% arara: pdflatex
% chktex-file 18
% chktex-file 8
\documentclass[a4paper,10pt]{scrartcl}
\usepackage{midgard}
\usepackage[utf8x]{inputenc}
\usepackage[T1]{fontenc}
\usepackage[ngerman]{babel}
\usepackage{verse}
\usepackage{siunitx}

\def\drugname{Basilisk}

\title{\drugname}
\author{Dimitri Scheftelowitsch}

\begin{document}

\maketitle

\begin{center}
  \midgardabenteuer{}
\end{center}

\begin{abstract} Dies ist ein stark an „Snow Crash” angelehntes
  Abenteuer. Der Autor bedankt sich bei Neal Stephenson für die Idee,
  für den einen oder anderen Charakter und bei Carsten Burgard für den
  Vorschlag, sich aus Büchern zu bedienen. Das Abenteuer kann ohne
  Verwendung eines bestimmten Regelwerks verwendet werden, impliziert
  aber eine Midgard-basierte Welt und nutzt Midgard-spezifische
  Begriffe, insbesondere, was NPCs angeht. Auch die Werte der NPCs sind
  für Midgard angegeben; falls man in einem anderen System spielt,
  sollte man sie entsprechend anpassen.
\end{abstract}

\section{Meta}

\paragraph{Danksagungen}

Der größere Dank gebührt Carsten Burgard für die Motivation, ein
Abenteuer zu schreiben sowie für hilfreiche Hinweise beim Entwurf und
bei der Nachbereitung. Ferner bedankt sich der Autor bei den
(freiwilligen) Testern Colin, Florian, Marcel und Qi. Die Idee des
Abenteuers ist aus dem Buch „Snow Crash” von Neal Stephenson
entnommen; das Setting kommt aus der Welt des Rollenspielsystems
\emph{Midgard}.

\paragraph{Getestete Systeme}
Das Abenteuer wurde in einer vereinfachten Form im System \emph{Dungeon
World} getestet. Der Autor glaubt, dass \emph{Dungeon World} für
detektivische Abenteuer, die spezifische Fähigkeiten wie
\emph{Wahrnehmung} oder \emph{Suchen} benötigen, wenig geeignet ist. Ein
Test in \emph{Midgard} steht noch aus.

\section{Einführung}

\paragraph{Kurze Zusammenfassung}

In einer Stadt in Moravod passieren Schlimme Dinge. Einige Menschen
verhalten sich auffallend komisch, aber es fällt nichts genaues auf, an
dem sie diese Anomalie feststellen können. Es ist vieles einfach nur…
anders. Vor allem die Magier scheinen ausgesprochen unfähig, benebelt
oder einfach nur vergesslich zu sein; manche Magier verschwinden. Auch
normale Bürger verhalten sich bei sorgfältiger Betrachtung sehr
generisch. Die Abenteurer müssen herausfinden, was dahinter steckt.

\paragraph{Spoiler für den Spielleiter}

Hinter dem Ganzen steckt ein Weltherrschaftsplot im kleinen Maßstab.
Ein reicher Händler hat vor, zum Bürgermeister von Saragin gewählt zu
werden.  Dabei setzt er nicht auf politische, sondern auch auf magische
Technologien. Insbesondere beauftragt er einen schwarzen Hexer mit einem
Zauber, das für eine beschränkte Zeit einen Menschen unter Kontrolle
bringt. Der Hexer wiederum sieht das als Anlass, einen Kult in der Stadt
zu etablieren. Mit den Ressourcen des Händlers wird ein für sich allein
interessantes Artefakt, das \emph{Buch der Gesetze} angeschafft. Es ist
uralt und gilt als verschollen (und das ist auch gut so); es ist eine
der ersten vollständigen Anleitungen zur Geistesmagie, inklusive
schwarzmagischer Praktiken. Das Abenteuer beginnt eine bis zwei Wochen
vor der Wahl des Bürgermeisters. Die genaue Deadline ist dem SL
überlassen, je nachdem, ob er mit viel oder wenig Zeitdruck spielen
möchte.

\section{Szenen}

\subsection{Warm-up: Hey… wer seid ihr eigentlich?}

Damit das Abenteuer losgehen kann, sollten die Spieler sich
kennenlernen. Da es vollkommen langweilig ist, sich in einer Taverne zu
begegnen und vollkommen motivationsfrei in ein Abenteuer zu fallen, gibt
es eine optionale Szene der Bekanntmachung.

\paragraph{Als Einstiegsabenteuer}
Ein möglichst seriös und nicht kämpferisch aussehender Abenteurer (zur
Not auch ein Dieb) wird von seinem aktuellen Arbeitgeber (Gilde,
Priester, …) beauftragt, einen relativ sensiblen Brief zu übermitteln.
Dabei wird er oder sie in einer dunklen Gasse nicht allzu weit vom Ziel
entfernt von bewaffneten und sehr unfreundlichen Typen begegnet. Dem
Spieler sollte klar sein, dass seine Auftraggeber die Worte „gestohlen”
oder „gewaltsam entrissen” nicht kennen, wohl aber „verloren” und
„selbst schuld”. Er sollte versuchen, um den Brief zu kämpfen, als wäre
es sein eigenes Leben (was auch wirklich stimmt); in jedem Fall werden,
als die Situation fast schon hoffnungslos erscheint, die anderen
Abenteurer in die Gasse gelenkt. Die Angreifer sind erschrocken und
verschwinden, wenn sie genug vom Kampf haben.

\paragraph{Überleitung aus einer Kampagne}
Alternativ stammt der Brief von einem mächtigen Magier (etwa Väterchen
Mischka, aber Duregg ist auch denkbar). Letzterer ist an einem magischen
Spiegel interessiert und schickt die Abenteurer nach Saragin, um
selbigen abzuholen.

\begin{textbox}
  {\bf Der Brief}

  Lieber Karl!

  Hiermit schicke ich Dir den von Dir gewünschten Stein, als Zeichen
  meiner Freundschaft und meines Respekts. Im Gegenzug nehme ich es mir
  heraus, Dich nach einem deiner Spiegel zu fragen.

  Dein \emph{(Name)}.

  \emph{Anlage: Stein der Macht, nur vom Empfänger benutzbar}
\end{textbox}

\subsection{Eröffnung: Viel Verwirrung um… nichts?}

Der Brief oben ist an Karl Petrov (s.u.) gerichtet, der ein Magier in
der Magiergilde ist. Er empfängt die Abenteurer je nach Tageszeit in der
Magiergilde oder bei sich daheim, ist aber nicht ganz bereit, das
magische Spiegelpaar herauszugeben; es ist nämlich noch nicht ganz
fertig.

\bestiarium[name=Karl Petrov, mag=true, typ=Mensch, grad=6,
In=m95, LP=20, AP=80, zauberEW=18, bes={Sb 10, Wk 70}, St=55, Gw=45]{}

Als kleine Entschädigung lädt er die Abenteurer in \emph{Die Insel} ein.
Kurz vor dem Eingang wird einer der Abenteurer von einer zwei Meter
hohen Gestalt, dem Eisbären (s.u.) begegnet, der \emph{\drugname{}}
anbietet. Karl nimmt es dankend (Sb ist hinreichend niedrig) an und
nimmt eine Schriftrolle in die Hand. Weitere Schriftrollen gibt es noch
nicht, aber der Eisbär ist gerne bereit, noch welche nachzuliefern,
falls Bedarf besteht.

\bestiarium[name=Bar, mag=false, typ=Mensch, grad=6,
In=m83, LP=34, AP=100, bes={Ko 81, Gs 70}, St=97, Gw=57]{}

Nach einer netten Unterhaltung in der \emph{Insel} erinnert sich Karl
an die Rolle und schaut sie sich an. Dies ist ein Text in Zauberschrift,
welcher einen Rauschzauber zu enthalten scheint, aber in Wahrheit ein
Geisteszauber ist, dessen Wirkung unter genauer erläutert ist. Die
Wirkung nach dem Ausrollen der Textrolle sieht wie folgt aus:

Die Rolle bleibt in der Luft hängen, als wäre sie auf Fäden aufgehängt.
Der Text verformt sich, die Zauberschrift vergröbert und verändert sich,
so, als würde man das Pergament mit sehr klein geschriebenem Text
schnell durchrollen.  Die Außenstehenden können das nur mit Glück
erkennen, mit einer W'keit von 50\% sehen sie nur, dass der Text sich
verändert und die Rolle sich komisch bewegt. Dann verschwindet das
Pergament.

Nach den ersten zwei Minuten passiert nichts, dann fängt der Zauberer
an, sein Gesicht auf eine unnatürliche Weise zu verziehen und fuchtelt
mit den Händen herum. Es sieht so aus, als wolle er schreien, aber es
gelingt ihm nicht. Dazu fängt er an, verschiedenfarbig zu leuchten,
worauf ihn die anwesenden Türsteher dann aus der \emph{Insel} werfen.
Das letzte, was die Abenteurer hören, ist: „Das Tor zum Geist… So alt
und doch so mächtig…”. Karl ist bis zum Ende des Abenteuers nicht mehr
ansprechbar. Der Eisbär ist zu diesem Zeitpunkt schon verschwunden. 

\subsection{Anhaltspunkte und Ermittlungen}

\begin{itemize}
  \item Politische Situation: Es sind Wahlen fällig. Es werden am
    zweiten Tag nach der Anreise der Spieler Wahlplakate ausgehängt, in
    denen man zur (Beobachtung der) Wahl aufgefordert wird.
  \item Gerüchte um einen Schwarzen Kult™: Irgendwo im Wald, munkelt
    man, gibt es einen Kult. So genau weiß man es nicht, aber der eine
    oder andere Jäger oder Pilzsammler wird sich an ominöse Gesänge
    erinnern. Tatsächlich liegt das daran, dass die Infizierten zum
    alten Sägewerk konvergieren und da zombiehaft vor sich
    herumexistieren und Rasputin anbeten. Ihre Gebete klingen
    grauenhaft. Damit hängt auch zusammen, dass die Anzahl der
    Obdachlosen, Bettler und anderer Vertreter der Unterschicht zum
    Beginn der zweiten Woche des Abenteuers auffallend abnimmt. Am
    Anfang kann man das nur merken, wenn man die weniger wohlhabenden
    Viertel persönlich und genau beobachtet.
  \item Die Magier werden auch ab einem bestimmten Zeitpunkt lahmgelegt.
    Dies übernimmt Bar mit Visionsrollen.
  \item Die Bibliothek der Magiergilde (man findet, wenn man gut sucht,
      ein altes Manuskript „Die Magie des Sehens” sowie ein noch älteres
      Manuskript „Zur neuen Zauberschrift”, in der die „neue”, für alle
      anwesenden Magier schon veraltete und reformierte Zauberschrift,
      hemmungslos kritisiert wird. Die „natürliche” Zauberschrift, die
      auch als die zu den Ursprüngen der Magie nähere proklamiert wird,
      sieht genau so aus, wie die auf der magischen Schriftrolle.)
  \item In „Die Magie des Sehens” stehen einige Zauberformeln, die mit
    intrinsischer Magie eines Lebewesens arbeiten, die ein gegebenes
    verzaubertes Objekt sehen.
  \item Falls nichts hilft: „Die Augen sind das Tor zum Geist, doch nur
    der Weise kann sie durchdringen” wird in einem Theaterstück, einem
    Buch, oder einem anderen kulturellen Artefakt explizit erwähnt. Zur
    Not singt auch der Barde Vitali Pripyat ein entsprechendes Lied.
  \item Bar hat eine sehr markante Tätowierung am Arm, waelische Runen
    mit dem Text \emph{Eingeschränkte Selbstkontrolle}. Für erfahrene
    Reisende ist das ein Zeichen, dass er schon mal in Birka straffällig
    war. (Rauferei mit einigen schwer verletzten)
\end{itemize}

\subsection{Motivation für PCs}

\begin{description}
  \item[Humanismus] Karl Petrov oder seine Familie (Frau: Magda,
    Tochter: Natascha) sind den Spielern hinreichend sympathisch, weil
    sie hinreichend viel Interaktion mit diesen NPCs hatten.
  \item[Verpflichtungen] Einer der Spieler hat immer noch einen Auftrag
    offen, und ohne Karl funktioniert das alles leider nicht.
  \item[Quest] Die Magiergilde setzt ein Preisgeld für die Auflösung der
    doch sehr sonderbaren Ereignisse aus.
  \item[Ein Angebot] Der Bürgermeister bekommt allerspätestens in der
    zweiten Woche des Abenteuers Wind davon, dass irgendwas nicht in
    Ordnung ist. Nun gibt es viele verschiedene Gründe, die dafür
    verantwortlich sein können -- aber hier sind ein paar Reisende
    angekommen, die etwas wissen können oder bei Ermittlungen die Rolle
    das Kanonenfutters spielen können. Für solche Zwecke hat er einen
    kleinen Keller unweit des Marktplatzes eingerichtet, wo er plausibel
    abstreitbar Angebote machen kann, die man nicht ablehnen kann.
  \item[Ein Angebot in schlimmer] Falls die Abenteurer mit ihren
    Ermittlungen nicht besonders vorsichtig sind, werden sie vom
    Bürgermeister, der sie für Verursacher des Chaos hält, konfrontiert
    und angeschuldigt. Ihnen werden wenige Tage gegeben, um ihren Ruf
    wiederherzustellen.
\end{description}

\subsection{Der Zauber}

\zauber[name=\drugname{},
  stufe=7,
  art=Wortzauber,
  prozess=Verändern,
  agens=Feuer,
  reagens=Feuer,
  Wz=Geist,
  Wb=Person,
  Wd={4W6 Tage},
Ur={dämonisch}]{}

\emph{\drugname{}} wirkt nicht nur auf Magier und existiert in mehreren Formen.
Für Magier, die die Zauberschrift lesen können, kann er in der Rolle
wirken. Für nicht magisch begabte Menschen wird er in Form eines
magischen Gifts oder einer Droge (Wirkstoff gemischt mit irgendwelchen
Pilzen) vorkommen. Einen Tag nach der Infektion fängt der Kranke an, den
Willen zu verlieren und driftet in Richtung des Waldes. Magier fangen
an, farbig zu leuchten und fallen ins Koma.

Hergestellt wird \drugname{} in Drogenform aus dem Blut eines Menschen, der
mit \drugname{} verzaubert wurde, Lotusblüten, Pilzen und Gold. Dazu ist die
kleine Kammer in Rasputins Laden ganz praktisch, und die Abenteurer
werden da auf jeden Fall relevante alchemische Spuren finden.

Die genaue Mechanik funktioniert wie folgt.

\paragraph{Infektion}

Ein nicht-Magier kann sich durch das Einnehmen der Droge infizieren.
Menge ist nicht wichtig, denn der Wirkstoff saugt sich über die
Schleimhaut ein.

Ein Magier kann durch das Lesen der Rolle infiziert werden. Hier ist
alles etwas komplizierter. Kritisch für die Infektion ist das
Verständnis des Inhalts zusammen mit der inneren Magie des Subjekts. Auf
der Rolle steht in Zauberschrift die alchemische Formel der Droge und
ihre Zusammensetzung. Falls der Inhalt nicht verstanden wird, passiert
nichts. Falls der Inhalt zu einem Teil gelesen und verstanden wird,
passiert nichts, aber man erfährt auch nicht viel. Falls der Inhalt von
einer nichtmagischen Person gelesen wird, kann diese den Inhalt
verstehen.

\paragraph{Wirkung}

Für nicht-Magier ist die Wirkung anfangs ähnlich zu einem leichten
Alkoholrausch. Alles ist gut, man hat das warme Bauchgefühl, das
Richtige zu tun, und man hört Stimmen, die sagen, was man tun muss. Bei
Magiern ist der berauschende Effekt stärker (magische Drogen haben auch
interessante Effekte), kann aber dazu führen, dass die Person einfach zu
nichts mehr in der Lage ist, mit leicht betrübten Augen herumsteht und
vielleicht anfängt, bunt zu leuchten.

\subsection{Der Kult im Wald}

Im Wald, etwa zwei Stunden von der Stadt entfernt, stand ein altes
Sägewerk.  Nun wird es nicht mehr benutzt, und der Weg dahin ist für die
meisten unbekannt. Bragin hat dieses Sägewerk früher gefunden, jetzt
befindet sich da Rasputins Kult.

Im Grunde ist der Sinn des Kults ziemlich primitiv: Opfergaben stärken
Rasputins Zauberkraft (in Midgard: AP), außerdem stellen zombifizierte
Bürger eine mittelmäßig organisierte Kraft dar, die Gegner auch
überrennen kann.

\paragraph{Das Buch der Gesetze}

Das Abenteuer ist insofern einfach gestrickt, als dass es ein relativ
offensichtliches Artefakt gibt, über das es gelöst werden kann. Da das
Buch der Gesetze eine Art Rootkit für die menschliche Seele ist, lassen
sich mit \drugname{} Infizierte über eine einfache Sprache der Gesetze
steuern. Andererseits sind Menschen in diesem Zustand nicht zu kreativer
Arbeit fähig.

Es lässt sich darin auch ein Befreiungszauber finden, der in der Sprache
der Gesetze formuliert ist.

Das Buch an sich ist eher ein Lehrbuch über Geistesmagie, geschrieben
von einem kompetenten und zeitlich nicht eingeschränkten Hexer. Der
Verfasser hält sich anonym, aber, wenn man es wirklich genau wissen
möchte, wurde es in KanTaiPan von einem sphärenreisenden Magier
geschrieben, der bei Dämonen zu Gast war und das gewonnene Wissen
dokumentiert hat. Bei Dämonen sind geistige Viren nicht nur bekannt,
sondern werden auch aktiv benutzt. Man kann eine Feuer-Metapher
anwenden: Wenn man weiß, wie man damit sicher umgeht, ist das vollkommen
akzeptabel, wenn man aber ahnungslos ist, ist man der Magie
ausgeliefert. Der besagte Magier hat geglaubt, das Wissen geheimhalten
zu können, wurde aber von Bragin überlistet. Nun ist das Buch entwendet
und in den falschen Händen.

\paragraph{Alternative Lösung}

Da es nicht so einfach ist, an das Buch der Gesetze dranzukommen, sollte
auch ein weniger direkter Weg auch zum Erfolg führen können. Allerdings
sollte man etwas mehr Zeit in Ermittlungsarbeit investieren. Eine alte
Alchimistin kann den Abenteurern eine Legende aus dem fernen KanTaiPan
erzählen, nach der der Gelbe Herr für die Menschen 88 Gesetze
geschrieben hat, den sie gefolgt haben. Nach der Legende ist der Sohn
des Gelben Herrn, der Anarch gegen seinen Vater gegangen und in einer
Nacht-und-Nebel-Aktion ein weiteres Gesetz verkündet, nach dem die
Menschen zum selbstständigen Denken verpflichtet wurden. Als Strafe
wurde der Anarch von seiner Sphäre verbannt.

Diese Legende weist darauf hin, dass es ein Gegenzauber gibt. In der Tat
gibt es in der Bibliothek moderne Abhandlungen über kantaipanische
Mythen und vielleicht auch noch das eine oder andere Bild. Es findet
sich auch ein Bild mit 89 Zeilen in alter Zauberschrift. Entziffern
sollte etwas Intelligenz erfordern.

\paragraph{Bragins Haus}

Viktor Bragin wohnt mit Familie in einem relativ großen Haus am
Marktplatz. Das Haus ist bewacht, initial von Söldnern, später auch von
Infizierten. In seinem Büro kann man Dokumente zu Reisen in KanTaiPan
(Münzen, Karten) finden. Seine Buchhaltung ist eher für die
Steuerverwaltung interessant. Für die Abenteurer können Holzspäne vom
Sägewerk und Einkaufslisten mit spezifischen alchemischen Inhalten
interessant sein.

\paragraph{Rasputins Antiquitätenladen}

Der Laden ist fast vollständig legal und unauffällig. Einen direkten
Verdacht kann nicht viel erwecken, aber bei genauer Betrachtung können
den Abenteurern folgende Details auffallen.

\begin{itemize}
  \item Es wird nichts magisches verkauft. Dafür, dass Rasputin sich
    nicht magisch kundig gibt, ist das nicht durch Zufall plausibel
    erklärbar.
  \item Im versteckten Keller findet sich dafür allerhand magisches
    Material.  Das Buch ist in einer Nische versteckt, aber es gibt auch
    so genügend Spuren: Pergamente, Briefe zu anderen Hexern mit
    Andeutungen und Fragen, Notizen.
\end{itemize}

Um in den Keller zu kommen, muss man aber an dem Eisbär vorbeikommen. Ab
Tag 4 wird das Buch der Befehle am Sägewerk gelagert, was die Auflösung
etwas schwieriger macht.

\paragraph{Die Magiergilde}

Man ist verwirrt, aber wenn man vertrauenserweckend erscheint, werden die 
Magier hilfsbereit sein, auch wenn sie selbst nicht ganz genau wissen, wonach 
sie suchen sollen. Mit jedem Tag fallen immer mehr Magier in den Bann
von \drugname{} und es wird schwieriger, Informationen verbal zu erhalten.
Dafür wird es umso einfacher, diese aus der Bibliothek zu extrahieren.

\paragraph{Die Hexenjäger}

Die Stadt ist groß genug, um eine kleine Vertretung der Lichtsucher zu
beherbergen. Nun ist aber in diesem Ort schon eine längere Zeit nichts
passiert, und in dieser Vertretung langweilt sich eine Hexenjägerin mit
einem Lehrling. Da sie eine direkte Gefahr für Rasputin sind, werden sie
zuerst ausgeschaltet. Spätestens nach einem Tag werden sie von dem
Eisbären besucht, der ihnen einen „Brief” (in Wahrheit eine Rolle mit
\drugname{}) zustellt.

\paragraph{Ein geheimes Treffen}

In der Nacht nach der ersten Szene wird den Abenteurern eine Nachricht 
zugestellt. Sie ist hastig geschrieben und darin wird der zauberkundigste der 
Abenteurer im Morgengrauen an einem Friedhof um ein Treffen mit dem Absender 
gebeten. Sie wird nach Möglichkeit persönlich zugestellt.

Auf dem Friedhof erwartet sie die Ehefrau des kürzlich verfluchten
Magiers.  Sie ist gleichzeitig Bibliothekarin der Magiergilde und ist
sich relativ sicher, dass man einige Informationen finden kann; viel
mehr Hilfe kann sie aber nicht leisten, da sie Angst um sich, ihre
Familie und andere hat.  Zumindest aber haben die Abenteurer eine
Möglichkeit, über Wissen+Weisheit den Fall aufzudecken. 

Bei dem Treffen ist es wichtig, dass die Abenteurer keine Dummheiten
machen. Falls sie laut oder zu auffällig sind, ist das Treffen
abgesagt, auf dem Treffpunkt liegt eine Nachricht „Vielleicht sollten
Sie etwas vorsichtiger sein”.

\paragraph{Ein Konzert}

Für kulturbegeisterte Abenteurer gibt es ein kleines Schmankerl, denn
gerade tritt im Goldbarren (s.u.) der Barde Vitaly Pripyat auf. Er ist
durch Titel wie „Mein Magierstab klemmt”, „Zauberschmelze”, „Kühl
bleiben”, „Tsunami” oder „Mein Herz ist ein brennendes Loch in der Erde”
bekannt. Bei diesem Konzert hat man die Gelegenheit, den Eisbären zu
sehen, der gerade \drugname{} in Drogenform an nichtsahnende
Kulturkonsumenten verteilt.

\paragraph{Der Fluch verbreitet sich}

Am dritten Tag stellen die PCs fest (sie können das auch früher sehen,
aber spätestens dann wird ihnen das explizit mitgeteilt), dass die
Unterschicht irgendwohin verschwindet. Es gibt weniger Bettler, der
Marktplatz ist nicht so laut wie üblich und die üblichen Nicht
Anständigen Orte sind relativ leer. Wenn man nachfragt, wird gesagt,
dass viele seit dem letzten Abend nicht gekommen sind, man hat sie nicht
mehr gesehen. Falls man zielgerichtet nachfragt, wird gesagt, dass viele
sich mit dem tollen \drugname{} zugedröhnt haben und den Rausch
ausschlafen wollten. Einen auffällig großen Typen (den Bären, wen denn
sonst?) hat man auch gesehen, er hat \drugname{} verteilt.

\subsection{Die Wahl des Bürgermeisters}

Zwei Wochen nach Beginn des Abenteuers wird gewählt. Der Wahlmodus ist einfach: 
Alle Mitglieder des Stadtrats stimmen in einem öffentlichen Verfahren auf dem 
Marktplatz ab. Es darf in der Theorie jeder Bürger der Stadt mit den üblichen 
Einschränkungen gewählt werden. Das ist auch die Lücke, die Bragin ausnutzen 
möchte.

Im besten Fall ist das die abschließende Szene mit Action. Bragin wird
in jedem Fall versuchen, den Stadtrat durch den Zauber handlungsunfähig
zu machen, im schlimmsten Fall hat er ein paar loyale Söldner, mit denen
er versuchen wird, den Stadtrat zu beeinflussen.

\paragraph{Option A:\@ Alles läuft nach Plan}
Falls der Zauber nicht gebannt ist, wird Bragin die von ihm verzauberten
Mitglieder des Stadtrats dazu bringen, ihn zu wählen. Er ist
vergleichsweise selbstsicher und wird außer den Zombies vielleicht einen
oder zwei Söldner stellen; die anderen werden sein Haus bewachen.

\paragraph{Option B:\@ Versenkte Mittel}
Falls die Abenteurer hinreichend viel Aufmerksamkeit erregt haben (aber
den Fluch nicht vollständig gebannt haben), kümmert sich Bragin um
Verstärkung im Form von (Anzahl Abenteurer $\times 2$) Söldnern; im
allergrößten Notfall wird er mehr oder weniger subtil Geiseln nehmen.

Falls Bragin hinreichend effektvoll außer Dienst gesetzt wird, werden
die Abenteurer entsprechend von den ehrenwerten Bürgern der Stadt
entlohnt (magische Waffen, Ruf, was auch immer die lokale Produktion
hergibt). Falls das nicht passiert, Bragin zum Bürgermeister erklärt
wird und die Abenteurer noch lebendig sind, verwandelt sich der Auftrag.
Stahlhand trifft dann die Abenteurer in den folgenden Stunden und
verkündet ihnen: "`Ich weiß nicht, wie das passiert ist. Es ist mir auch
nicht so wichtig. Ich weiß, dass mein Stadtrat dazu gezwungen wurde. Ich
nehme an, dass ihr etwas damit zu tun habt. Ich schlage euch vor, euren
Fehler zu korrigieren. Noch diese Nacht soll Bragin von der Bildfläche
verschwinden. Ihr am besten auch. Vergesst nicht, dass wir ab jetzt in
einem Boot stecken."'

\paragraph{Option C:\@ Hey! Wo ist die Action?}
Falls die Abenteurer es überraschenderweise schaffen sollten, Bragin und
Rasputin einzeln zu besiegen, verläuft die Wahl relativ unspektakulär.
Dafür dürfen die Helden alles looten, was sie finden.

\subsection{Abschluss}

Falls alles gut geht und die Bösewichte neutralisiert wurden, lädt der
noch oder schon wieder amtierende Bürgermeister die Abenteurer
(+Helfer-NPCs) zu einem Gespräch ohne weitere Zeugen ein. Er fragt sie
nach den Details aus, und bietet, erfreut über den glücklichen Ausgang
der Ermittlungen, Belohnungen an:
\begin{itemize}
  \item magische Waffen (nichts außerordentliches, aber nichts, was man
    einfach so kaufen kann),
  \item Ermittlungen im Auftrag des Bürgermeisters (für potenzielle
    Folgeabenteuer),
  \item Geld (jeweils 70GS pro Abenteurer),
  \item Empfehlungsbriefe
\end{itemize}

\section{Stadt}

\subsection{Hintergrund und Bewohner}

Das Geschehen findet in Saragin statt. Saragin ist eine Stadt mit 900
Einwohnern, von denen eine signifikante Anzahl (200) Zwerge sind. Die
Stadt lebt von Tagebau und der damit verwandten Schwerindustrie, so
schwer, wie man sie auf Midgard halt finden kann. Seine spirituellen
Nachbarn in anderen Sphären sind Vorkuta, ehem. Cardiff, ehem.
Saarbrücken und Norilsk, das Stadtbild ist nicht ganz unähnlich. Die
Meistereien der Zwerge rußen freundlich in den Himmel, das einladende
Hämmern und Klopfen kann man in jeder Ecke der Stadt gut hören. Die
Stadt baut Kohle, Metalle und interessante Kristalle ab und wandelt sie
in Schwerter, Schilde, magische Artefakte und ab und an Haushaltswaren
um.

Die Geschichte der Stadt ist lang. Die Zwerge haben hier schon sehr
lange gewohnt, doch die Menschen, die (aus geologischer Sicht) recht
frisch in diesen Gebieten sind, haben das Leben hier eher bereichert als
umgekehrt. Viele der Bewohner sind verurteilte Verbrecher, die ihre
Sünden in den Minen kompensieren mussten (oder noch müssen). Faktisch
sind das gar nicht (mehr) so viele, aber die in diesen Kreisen
gewöhnliche Umgangsart hat sich mehr als notwendig in das Leben der
Stadt eingeprägt.

Die Bewohner der Stadt sind
\begin{itemize}
  \item 200 Zwerge: Tagebau, Industrie, Hafenverwaltung
  \item 650 Menschen: Handel, (Zwangs-) Arbeiter, Magier
  \item 50 andere: Durchreisende Händler, der eine verlaufene Elf, …
\end{itemize}

\subsection{Orte (nicht notwendigerweise plotrelevant)}

\subsubsection{Gaststätten}

\begin{description}
\item[Das Einhorn] ist das Sinnbild für die ranzigste, abgewrackteste
  Kneipe auf Midgard. Keine nur im Geringsten anständige Person möchte
  nicht einmal in einem gemeinsamen Kontext mit diesem Lokal gefunden
  werden, geschweige denn von physikalischer Nähe. Es treiben sich da
  weitgehend Kleinkriminelle, Bettler und andere Menschen und
  menschenähnliche Gestalten herum, die von den feinen Manieren nicht
  entstellt wurden. Je nach Tageszeit kann eine Schlägerei entstehen;
  das schlimmste jedoch sind lokale missglückte Gestalten, die sich für
  Barden halten und traurig anmutende Lieder über das harte Los eines
  Gefangenen spielen, die musikalisch und textuell unter aller Sau sind.

  \begin{verse}
    Das Leben ist hart \\
    Wie meine Gedichte \\
    Doch im Herzen bin ich zart \\
    Wie der Turanerin ihre Brüste\footnote{Der Autor möchte sich hiermit
    ausdrücklichen von den in diesem Gedicht lyrischen Mitteln
  distanzieren.}
  \end{verse}

\item[Zum Becher] ist eine klassische Arbeiterkneipe. Da kann man im
  Grunde nette und freundliche, wenn auch manchmal sehr einfach
  gestrickte Menschen finden, die zwar für abstruse Unterhaltung sorgen
  können, aber im Durchschnitt ziemlich normal sind. Es kommen so
  Sprüche wie „Diese Halblinge, was wollen sie alle hier?” und „Nein,
  verstehe mich nicht falsch, mein bester Freund ist Elf…”

\item[Der Goldbarren] ist ein mehrheitlich zwergisches Lokal, es
  empfängt aber auch Menschen. Es besticht durch die glänzenden
  Schilder, nahrhaftes Essen und gutes Bier, schreckt aber durch relativ
  hohe Preise ab. Da die meisten Zwerge in der Industrie arbeiten, ist
  das für sie weitgehend in Ordnung, für einen Wanderzirkus können die
  Preise auch abschreckend wirken.

\item[Das goldene Lamm] wird von den wohlhabenderen Bewohnern gerne
  frequentiert.  Hohe Qualität, hohe Preise und das eine oder andere
  kulturelle Phänomen, falls welche diese Stadt im Norden besuchen.
  Kanonischer Ort, in dem man feiert oder auch mit den wichtigen Leuten
  der Stadt zu Abend isst.

\item[Die Insel] zeichnet sich dadurch aus, dass sie weder Schilder noch
  sonstige Aushänge hat und von außen nicht als ein Lokal
  identifizierbar ist. Hier kann man sich ungestört vom Leben der Stadt
  und dem alltäglichen Stress erholen und vielleicht die eine oder
  andere wichtige Verhandlung führen. Das Stichwort ist hier
  „ungestört”: Musik, Musikinstrumente, singende Barden, Animateure und
  professionelle Hochzeitsorganisatoren sollen draußen bleiben. Besitzer
  von Musikinstrumenten können sie abgeben, bei einem Versuch, laut zu
  singen wird man freundlich, aber bestimmt gebeten, zu gehen. Gerüchten
  zufolge hat man in nicht allzu lang vergangenen raueren Zeiten hier
  auch Kopfgelder für besonders laute Gäste ausgeschrieben; der Besitzer
  ist überhaupt nicht daran interessiert, diesen Gerüchten zu
  widersprechen. Jetzt zumindest ist das ein netter, nicht allzu teurer
  Ort mit ein paar Zimmern zur privaten Unterhaltung. Als Erinnerung an
  die guten alten Zeiten gibt es im Menü die Spezialität des Hauses
  „Bardenherz” (bei einer hinreichend guten Probe auf Schmecken enttarnt
  es sich als ein -- ziemlich gut gemachtes -- Rinderherz).
\end{description}

\subsubsection{Magische Infrastruktur}

Die \emph{Magiergilde} besteht aus einem mehrere mittelgroße Gebäude
umfassenden Komplex mit einem Innenhof. Sie verfügt über eine nicht
allzu kleine bibliothek und spezialisiert sich in angewandter Magie und
Thaumatographie (Herstellung magischer Waffen und, wenn man Zeit hat,
magischer Kleinteile.)

Im Moment befindet sich die Stadt vor einem leichten Wandel. Die Zeiten
der Zwangsarbeit sind in die Vergangenheit gerückt; die Industrie bringt
genug Geld ein, um auch an Dinge jenseits magischer Waffen zu denken,
und so viele Kriege gibt es im Moment auch gar nicht… Karl Petrov denkt
intensiv über „Befestigen”-Zauber nach, und der Fürst von Belogora hat
einer Reihe der lokal ansässigen Magier etwas Gold versprochen, wenn sie
einen Weg finden, magische Spiegel zu konstruieren.

\subsubsection{Sonstige Sehenswürdigkeiten}

Da die Natur eher kalt und unfreundlich ist, hat die Stadt keine
Stadtmauer. Weglaufen kann man nicht, und hin kommt man ohnehin nur,
wenn man es wirklich nötig hat.

\subsubsection{Das Sägewerk}

Das alte Sägewerk wird von Bragins Söldnern ($2 \times$ Anzahl der
Abenteurer) bewacht. Sie haben Langschwerter (EW 8) und Armbrüste (EW 7)
und keine Angst, sie einzusetzen. Je einer steht auf einem Wachturm, der
Rest ist über das Gebiet ($1 \times 1$\si{km}) verteilt.

\bestiarium[name=Bragins Södner, mag=false, typ=Mensch, grad=3,
In=m65, LP=26, AP=38, St=80, Gw=76]{}

\section{NPCs}

In diesem Abschnitt werden die für die Handlung wesentlichen
Nichtspielercharaktere beschrieben. Meistens beschränkt sich die
Beschreibung auf die Motivation und den jeweiligen Hintergrund des
Charakters.

\subsection{Viktor Bragin}

Viktor ist der wesentliche Antagonist. Seine normale Beschäftigung ist
der Handel, allerdings hat er durch das viele Geld, das er verdient hat,
einen Drang zu Macht entdeckt. Die Tatsache, dass das Fürstentum
Belogora faktisch von der Zentralregierung unabhängig ist, hat ihn zur
Idee verleitet, die Stadt Saragin unter den Nagel zu reißen, wenn schon
die völkerrechtlichen Verhältnisse nicht so ganz klar sind.

\paragraph{Motivation und Hintergrund} Macht, Geld. Glaubt relativ
ernsthaft, dass er Bürgermeister sein soll, weil er wirklich weiß, wie man
eine Stadt regiert und niemand anders diesen Posten wirklich verdient
hat. (Aktuell ist die Stadt zwar weitgehend gepflegt, aber ihm ist das
egal.) Hat „Fragen geklärt”, aus seiner Jugendzeit sind ein paar Narben
erkannbar. Es kursieren Gerüchte über seine initiale Anhäufung des
Kapitals, die verschieden blutig und sicherlich nicht alle falsch sind,
aber man kann die Wahrheit nur schwer extrahieren, da dies nicht in
Saragin geschah. Zu den Abenteurern ist er gütig und möchte einen
Eindruck eines guten Verwalters erwecken. Falls er mit der Wahrheit
konfrontiert wird, gibt er sich unverständlich; „als ob man in einer
Stadt mit einer Diebesgilde von Moral sprechen kann“. Falls er mit
Staatsgewalt überführt wird, wird er versuchen, zu fliehen. Vielleicht
wird er auch Erfolg haben, denn sonst haben die Lichtsucher und der
Fürst von Geltin/Belogora ein Problem mit ihm, im Notfall wegen
Steuerhinterziehung und Missachtung magischer Sicherheitsvorschriften…

\bestiarium[name=Viktor Bragin, mag=false, typ=Mensch, grad=5,
In=m85, LP=28, AP=40, St=75, Gw=70]{}

\subsection{Alexander Stahlhand}

Alexander Stahlhand ist der amtierende Bürgermeister der Stadt Saragin. 

\paragraph{Motivation und Hintergrund} Macht, Geld, aber in etwas
anderen Maßstäben. Er regiert diese Stadt schon seit 15 Jahren und ist
in dieser Zeit etwas entspannt und ruhig geworden. Zumindest gibt er
sich wie eine entspannte Vaterfigur.

In Wahrheit ist er ein administrativ-politisches Tier mit weit
reichender Erfahrung und noch weiter reichendem Gedächtnis. Er kennt
fast alle Entscheidungsträger dieser Stadt und spielt auf ihren
gegenseitigen Konflikten wie Frank Zappa auf seiner Gitarre. Die
Tatsache, dass jemand gegen ihn kandidiert, kann er nicht erklären,
fühlt aber, dass die Erklärung ihm nicht gefallen wird. Im Notfall kann
auch er den PCs ein Angebot machen, das sie nicht ablehnen können
werden.

Auch über ihn kursieren Gerüchte aus seiner Jugend. Die ist zwar etwas
länger her, aber damals war er als Sohn eines Händlers in das
Abenteurergeschäft eingestiegen. Nicht alles, was er erlebt hat, war
großartig, aber vieles brachte ihm Gewinn und Bekanntschaften.
Irgendwann entschied er sich für einen Karrierewechsel in Richtung
Administration und Verwaltung.

\subsection{Stanislav Rasputin}

Der zweite Antagonist des Abenteuers is ein Schwarzhexer, der nach
Wissen und Macht strebt, aber gleichzeitig versucht, in einer doch eher
magisch dicht besiedelten Stadt nicht aufzufallen. Daher gibt er sich
als Alchimist und Antiquitätenhändler, der für die magischen Bewohner
uninteressant und für magische Reisende unauffällig ist.

Er ist in seinen 30ern, sieht durchschnittlich aus und hat einen
durchdringenden Blick, der schon viele Besucher verjagt hat. Er verfügt
über eine Reihe von Zaubern, unter Anderem \emph{Angst}, \emph{Blitze
Schleudern}, \emph{Wahnsinn}, \emph{Bannen von Zauberwerk} u.\,A.

\paragraph{Hintergrund} Ein relativ üblicher 08/15-Schwarzhexer. Dadurch
interessant, dass er in einer Stadt lebt und bisher unauffällig war. Der
Grund dafür ist einfach: Bis zur Entdeckung des Buchs der Gesetze hat er
sich auf Nachforschungen und Suche nach Alliierten konzentriert.

\bestiarium[name=Stansilav Rasputin, mag=true, typ=Mensch, grad=6,
In=m97, LP=21, AP=78, zauberEW=18, St=45, Gw=65]{}

\subsection{Der Eisbär (Bar)}

Bar ist ein (fast) normaler Nordlandbarbar, der aus vielen verschiedenen
Gründen sich nicht mehr im Nordland aufhält. Seit er in Saragin
aufgeschlagen ist, hat Rasputin ihn als Helfer und Mensch fürs Grobe
eingestellt. Wie zu erwarten, ist Bar groß und stark, aber auch für
Barbaren überdurchschnittlich intelligent.

\paragraph{Hintergrund} Bar ist in der Gegend um Birka herum
aufgewachsen und hat sich mit dem dortigen Geschäftsmodell nicht
anfreunden können. Hat mehr als genug Feinde gemacht, nach seiner
unfreiwilligen Abreise sind sehr plötzlich ein paar Häuser der dort
ansässigen Händler in Flammen aufgegangen. Mag deswegen Menschen und
Zivilisation nicht und ist mit den Methoden von Rasptuin/Bragin zu einem
Großteil einverstanden.

Er ist ein sehr starker NPC, was ihn zu einem sehr gefährlichen
Gegner macht. Er besitzt eine magische Waffe (Streitkolben mit Schaden
2W6+2, und Angriffsbonus +4, was zusammen einen Angriffswert von 18
ergibt), ist aber hinreichend vorsichtig, um sich nicht in ausweglose
Situationen zu verwickeln.

\paragraph{Für erfahrene Abenteurer} Zur weiteren Erschwerung ist er ein
Souverän. Ein Souverän in dem Sinne, als dass er einen Ring mit sich
trägt, in dem ein höherer Dämon gefangen ist. Es ist eine sehr dumme
Idee, diesen Dämon freizulassen; Dämonen ohne Zwingkreis sind an sich
sehr gefährlich, und dieser wird alles im Umkreis von 1km mit
\drugname{} verzaubern.

\paragraph{Karl Petrov}

Karl ist einer der kompetentesten Magier der einschlägigen Gilde. Seine
Kompetenz in Sachen Fernwirkung von Magie ist nur durch seine
Umstrittenheit übertroffen. Petrov ist recht impulsiv, kann sich zwar in
"`normalen"' Situationen zurückhalten, wird aber keiner Versuchung
widerstehen können. Eine dieser Versuchungen ist die Fernwirkung der
Magie, die ihn dazu veranlasst hat, ein Buch mit dem Titel „Sprechen mit
magischen Gegenständen” zu schreiben. Dies (und manche seiner anderer
Eigenheiten) sorgt für nicht allzu feindselige, aber manchmal ziemlich
grobe Lacher.

\end{document}
